\paragraph{Statement of principal finding}
        This study illustrated the implications of applying 
    combined strategies of lockdown and vaccination to mitigate
    the curse of COVID-19. Our experiments suggest that a 
    combined and well-balanced policy between the relaxing 
    times of lockdown and vaccine rollout would improve the 
    mitigation of COVID-19 symptomatic prevalence and deaths,
    but with a delicate balance whit the economic impact. 
    We obtained lockdown policies that modulate individuals' 
    release rate synchronized with the symptomatic prevalence 
    and vaccine rollout speed. Our vaccination policies captured
    time windows where it is convenient to modify the vaccine 
    administration according to the symptomatic prevalence and
    economic cost.
\paragraph{Strengths and weakness of the study}
    %TOPIC SENTENCE
    \begin{CheckList}{Goal}
        \Goal{open}{Argument for Strengths}
            \begin{CheckList}{Task}
                \Task{started}{Piecewise optimal policies}
                \Task{started}{Practical}
                \Task{started}{Modulation between
                    relaxation and inclusion of 
                    population in lockdown}
            \end{CheckList}
    \end{CheckList}

    \paragraph[]{STRENGTHS}
	        We got optimal policies that are practical. 
        Whit practical, we mean those capture actions that can be
	    implemented in the real world. Because the policies rely on 
	    piecewise constant values for a given period (here we use 
	    days), we argue that they are more realistic than policies 
	    that consider continuous functions that could implicate
	    continuous changes of a given action. For example, 
	    it is more realistic to administer a fixed number of jabs
	    along with a given date than following a configuration of
	    several doses that would change continuously on the same date.
    
            Besides, we synchronized the trade of between
        lockdown-release, vaccine-rollout but balancing the 
        economic cost.

    \paragraph[]{WEAKNESS}
        \begin{CheckList}{Goal}
            \Goal{open}{Argument for Weakness}
                \begin{CheckList}{Task}
                    \Task{done}{Vaccine Multi-dose.}
                    \Task{started}{Protection only against severe symptoms.}
                    \Task{started}{What expect respect 
                        to the protection against transmission.
                    }
                \end{CheckList}
        \end{CheckList}
  
            One limitation was the assumption of only one vaccine. 
        In almost all countries the vaccination campaigns consider at 
        least two developments. For example, in Mexico the vaccine portfolio 
        (at the day of writing) relies on at least four developments. Moreover, 
        this vaccine portfolio includes developments that differs in the number 
        of required doses. For example, Pfizers require two doses while Cansino 
        Bio only demands one.
        
            Further, we do not face additional vaccine administration 
        requirements as the time between doses or logistics\textemdash 
        each development implies different protocols. Since the approved 
        vaccine's protective efficacy against transmissions of SARS-CoV2 
        remains under study, we do not consider this hypothesis in our
        formulation. We recognize that this parameter would play an 
        important role, which is plausible to consider in future formulations.

    \paragraph{Strengths and weakness in relation to 
        other studies, discussing important differences
        in results
    }
    \todo{citations}
    \paragraph{STRENGTHS}
        \begin{CheckList}{Goal}
            \Goal{open}{Argument for Weakness}
                \begin{CheckList}{Task}
                    \Task{started}{%
                        NPI's optimal Policies with non Vaccination
                        }
                    \Task{started}{%
                         
                        and compare our results.}
                    \Task{started}{What expect respect 
                        to the protection against transmission.
                    }
                    \Task{open}{Optimal Control of Vaccination Rate
                    cite cite{Libotte2020} and compare our results
                    }
                \end{CheckList}
        \end{CheckList}
        
        Works like 
    \cite{Perkins2020,Palmer2020,Djidjou2020,Asamoah2021,Nabi2021} 
    developed NPI's with optimal control for COVID-19. Some 
    contribution of this list  combines two or more strategies to
    mitigate prevalence and deaths due to SARS-CoV-2. For example,
    \cite{Nabi2021} formulates a fractional ODE model with public
    education, treatment, and management of asymptomatic cases.  
    Assamoth in \cite{Asamoah2021} puts 
    special attention to the economic cost related to the combined 
    health protocol of physical distancing, media advocacy,  
    wearing a mask, hand-washing, lockdown, and contact tracing. 
    Similarly, \cite{Djomegni2021,Jiang2020,Ullah2020}
    reports optimal policies but with significant emphasis on
    other aspects related to the spread dynamics of COVID-19.
    However, the mentioned works developed policies based on
    continuous functions and did not optimize or not include 
    vaccination. In our opinion, these kinds of policies would 
    be impractical for decision-makers. 
	    
	    Here we argue that our policies give a precise sequence 
	of actions that are feasible for implementation. Further, we 
	calculated the cost and balanced its implementation 
	synchronized and balanced with the economic implications.

    \paragraph{WEAKNESS: Static optimization }
        \begin{CheckList}{Goal}
            \Goal{open}{Argument for Weakness}
                \begin{CheckList}{Task}
                    \Task{started}{Allocation.
                        Comment about optimal allocation as a 
                        static optimization problem and cite
                        cite {Bubbar, Buckner, Moore2021}.
                    }
                    \Task{open}{%
                        Other Approach cite and discuss 
                        cite{Lazembink2020}%
                    }
                \end{CheckList}
        \end{CheckList}
    
        On the other hand, another limitation of this work 
	was the lack of stratification and risk groups. Because the 
	year of life lost is susceptible to this regard, and the 
	productive sector of an economy is closely related to the 
	workforce, we might prioritize accordingly. However, we see 
	that our result could complement the prioritization policies 
	of relevant works like \cite{Bubar2021,Matrajt2020a}, Buckner2020].
	While optimal vaccine prioritization strategies answer 
	the question: Who to vaccine first, we answer when intensifying 
	vaccine rollout	and lockdown release. Thus our contribution 
	would be complementary.        

    \paragraph{Meaning of the study: possible explanations 
        and implications for clinicians and policymakers}

        New and more contagious variants of SARS-CoV-2 appeared, 
    and the COVID-19 vaccine supply would be scarce, slow, and 
    complicated for countries like Mexico. Thus we expect three 
    synchronized events: another COVID-19 wave, an
    intensification of the vaccine rollout, and other lockdowns. 
    Here we argued that the above strategies also must be 
    synchronized and consider a delicate balance with the 
    economic impact.

    \paragraph{Unanswered questions and future research}
        
        Despite that NPIs have been implemented in most countries 
    to mitigate COVID-19, these strategies cannot develop immunity. 
    Thus, vaccination becomes the primary pharmaceutical measure. 
    However, this vaccine has to be effective and well implemented 
    in global vaccination programs. Each development implies 
    particular issues--like logistics number of doses, 
    secondary effects, and others. For example, Mexico is 
    administrating vaccines from Pfizer-BioNtech, AstraZeneca, 
    CanSino-Bio, the Sputnik V from Russia, and others. 
    Each of these vaccines implies different requirements 
    for its management, provides protection with different 
    efficacies, and differ in its number of doses. We believe
    that this complicated landscape has significant implications
    in the design of health policies.

        Thus, new challenges in distribution, stocks, politics, 
    vaccination efforts, among others, emerge.     
    